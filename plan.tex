\documentclass{article}
\usepackage{amsmath}
\usepackage{graphicx}

\begin{document}

\tableofcontents
\newpage

\section{Introduction}
\begin{itemize}
    \item Contexte et importance des équations non linéaires en mathématiques et dans les sciences appliquées.
    \item Présentation de la méthode de Newton et son rôle crucial dans la résolution des équations non linéaires.
\end{itemize}

\section{Définitions et Concepts Fondamentaux}
\subsection{Théorème du développement de Taylor}
\begin{itemize}
    \item Présentation et explication du théorème de Taylor et son application à la méthode de Newton.
\end{itemize}
\subsection{Multiplicité de la racine d'une fonction}
\begin{itemize}
    \item Explication détaillée de la multiplicité des racines et son impact sur la convergence des méthodes itératives.
\end{itemize}
\subsection{Équation non linéaire}
\begin{itemize}
    \item Définition des équations non linéaires avec des exemples pratiques.
\end{itemize}
\subsection{Aperçu des méthodes de résolution des équations non linéaires}
\begin{itemize}
    \item Brève présentation de différentes méthodes (bissection, méthode de la sécante, etc.) avant de se concentrer sur la méthode de Newton.
\end{itemize}

\section{Exemples d'Équations Non Linéaires}
\subsection{Types d'équations non linéaires}
\begin{itemize}
    \item Présentation de différents types d'équations non linéaires (polynomiales, transcendantales, etc.).
\end{itemize}
\subsection{Méthodes de résolution alternatives}
\begin{itemize}
    \item Comparaison succincte des méthodes de résolution autres que celle de Newton (avec un exemple pour chaque méthode).
\end{itemize}

\section{Méthode de Newton}
\subsection{Détermination de la correction \(\delta x\)}
\begin{itemize}
    \item Dérivation et explication de la formule pour \(\delta x\) dans le cadre de la méthode de Newton.
\end{itemize}
\subsection{Algorithme de la méthode de Newton}
\begin{itemize}
    \item Présentation détaillée de l'algorithme, accompagnée d'un pseudocode.
\end{itemize}
\subsection{Interprétation géométrique}
\begin{itemize}
    \item Illustration graphique de la méthode de Newton pour une meilleure compréhension visuelle.
\end{itemize}
\subsection{Analyse de convergence}
\begin{itemize}
    \item Discussion sur les conditions de convergence et la rapidité de la méthode de Newton.
\end{itemize}
\subsection{Cas des racines multiples}
\begin{itemize}
    \item Explications des défis posés par les racines multiples et les solutions possibles pour améliorer la convergence.
\end{itemize}

\section{Implémentation en Scilab}
\subsection{Présentation du code Scilab}
\begin{itemize}
    \item Introduction à l'implémentation de la méthode de Newton en Scilab.
\end{itemize}
\subsection{Explication du code}
\begin{itemize}
    \item Décomposition du code en segments avec des explications détaillées.
\end{itemize}
\subsection{Application pratique}
\begin{itemize}
    \item Exemple concret d'utilisation du code Scilab pour résoudre une équation non linéaire.
\end{itemize}

\section{Conclusion}
\begin{itemize}
    \item Synthèse des principaux points abordés dans l'exposé.
    \item Discussion sur l'importance et l'efficacité de la méthode de Newton.
    \item Exploration des futures applications et des développements possibles de la méthode de Newton.
\end{itemize}

\section{Bibliographie}
\begin{itemize}
    \item Liste des livres, articles académiques et sites web utilisés.
    \item Suggestions de lectures supplémentaires et de ressources pour approfondir le sujet.
\end{itemize}

\end{document}
